\documentclass{article}\usepackage{amsmath,amssymb,amsthm,tikz,tkz-graph,color,chngpage,soul,hyperref,csquotes,graphicx,floatrow, listings}\newcommand*{\QEDB}{\hfill\ensuremath{\square}}\newtheorem*{prop}{Proposition}\renewcommand{\theenumi}{\alph{enumi}}\usepackage[shortlabels]{enumitem}\usepackage[nobreak=true]{mdframed}\usetikzlibrary{matrix,calc}\MakeOuterQuote{"}\usepackage[margin=0.75in]{geometry} \newtheorem{theorem}{Theorem}\newcommand{\Z}{\mathbb Z}\newcommand{\R}{\mathbb R}\newcommand{\Q}{\mathbb Q}\newcommand{\N}{\mathbb N}

\title{CS70 - Lecture 14 Notes}
\author{Name: Felix Su$\quad$SID: 25794773}
\date{Spring 2016$\quad$GSI: Gerald Zhang}
\begin{document}
\maketitle

%%%% Topic %%%%
\subsection*{Counting}
%%%% Notes %%%%
\textbf{Tree Counting: Slow}
\begin{itemize}
    \item Build up string by bits, total amount of leaves is total possibilities
\end{itemize}
\begin{mdframed}
\textbf{First Rule of Counting: Product Rule:}
\begin{itemize}
    \item If objects constructed from a sequence of choices $n_1, n_2, ..., n_k$
    \item Total number of objects = $n_1 \times n_2 \times \cdots \times n_k$
\end{itemize}
\end{mdframed}
\textbf{Counting Functions/Polynomials}
\begin{itemize}
    \item There are $|T|^{|s|}$ functions $f : S \rightarrow T$
    \begin{itemize}
        \item $|T|$ choices for mapping of $f(s_i)$ (Use product rule)
    \end{itemize}
    \item $p^{d+1}$ polynomials of degree $d \bmod{p}$
    \begin{itemize}
        \item $p$ choices for each of the $d+1$ coefficients
    \end{itemize}
\end{itemize}
\textbf{Permutations}
\begin{itemize}
    \item Derived from the first rule of counting (product rule)
    \item Choose from less items each step
    \item Permutations of $n$ objects: number of orderings of $n$ objects (no replacements)
    \begin{itemize}
        \item $n \times (n-1) \times (n-2) \times\cdots \times 1 = n!$
    \end{itemize}
    \item Number of one to one functions $|S| \rightarrow |S|$ 
    \begin{itemize}
        \item Decreasing choices every step: $|S| \times |S|-1 \times \cdots \times 1 = |S|!$
    \end{itemize}
\end{itemize}
\begin{mdframed}
\textbf{Permutation Formula}
\begin{itemize}
    \item Number of different samples of saize $k$ from $n$ numbers \textbf{without replacement}
        \begin{equation}nPk = n \times (n-1) \times (n-2) \times\cdots \times (n-(k-1)) = \frac{n!}{(n-k)!}\end{equation}
\end{itemize}
\end{mdframed}
\textbf{Counting Sets: When order doesn't matter}
\begin{mdframed}
\textbf{Second Rule of Counting: Order Doesn't Matter (Combination):}
\begin{itemize}
    \item If order doesn't matter, count the number of ordered objects (permutations) and divide by number of orderings
    \item Choose $k$ out of $n$ possibilities
    \begin{equation}\binom{n}{k}= nCk = \frac{n!}{k!(n-k)!}\end{equation}
\end{itemize}
\end{mdframed}
\begin{mdframed}
\textbf{Sampling:}
\begin{itemize}
    \item Sample $k$ items out of $n$
    \item Without replacement:
    \begin{itemize}
        \item If order matters (first rule): $\frac{n!}{(n-k!)}$
        \item If order does not matter (second rule): $\frac{n!}{k!(n-k!)}$
    \end{itemize}
    \item With replacement:
    \begin{itemize}
        \item If order matters (first rule): $n^k$
        \item \textbf{see Stars and Bars formula (3)}
    \end{itemize}
\end{itemize}
\end{mdframed}
\textbf{Anagrams:}
\begin{itemize}
    \item First rule on total number of letters $N$: $N!$ total permutations
    \item Divide by the number of duplicate permutations generated due to $D$ duplicate letters: First rule: $D!$
    \item total distinct permutations = $\frac{N!}{A!B!\cdots D!}$ (can have multiple duplicate sets of letters)
\end{itemize}
\textbf{Stars and Bars:}
\begin{itemize}
    \item Ways $k$ people split $n$ things
    \item Ways to add up $k$ numbers to sum to $n$
    \item $k$ undordered choices from set of $n$ possibilities
    \item $\dbinom{\textrm{total} + \textrm{(sections - 1)}}{\textrm{sections - 1}}$
    \begin{equation}\binom{n+k-1}{k-1}\end{equation}
\end{itemize}
%%%% Topic %%%%
\subsection*{Summary}
%%%% Notes %%%%
\textbf{First Rule (Product)}
\begin{itemize}
    \item $k$ samples
    \item With replacement: $n^k$
    \item Without replacement: $\frac{n!}{(n-k)!}$
\end{itemize}
\textbf{Second Rule (Division)}
\begin{itemize}
    \item When order doesn't matter (sometimes): can divide
    \item Without replacement (order doesn't matter): $\binom{n}{k}=\frac{n!}{(n-k)!k!}$ $n$ choose $k$
    \begin{itemize}
        \item You pick a different object every time. The total amount of orderings for your $k$ objects is $k!$, so divide sample without replacement by $k!$ because order doesn't matter
    \end{itemize}
\end{itemize}
\textbf{One-to-one Rule}
\begin{itemize}
    \item Equal in number if one-to-one (Bijection)
    \item With replacement (order doesn't matter): $\binom{k+n-1}{n-1}$
\end{itemize}
\end{document}